% CREATED BY DAVID FRISK, 2015
\chapter{Methodology}
\lettrine[findent=2pt]{\fbox{\textbf{T}}}{ }his chapter is divided in three parts: Data collection, Analysis of data, Visualization of the important components of electrical architectures for stakeholders and The evaluation of the visualization process.
%
\section{Data collection}
On collecting the necessary data, the team firstly had to think carefully on what are the goals that we have in this thesis, then we had came up with questions for interviewing the stakeholders, after that we identified the groups of stakeholders which had impact on this project. The next step we did after identifying stakeholders was identifying their needs. We also needed the data from the software Elektra, the data from the stakeholder and the data from Elektra were the most important ones to facilitate the development of the visualization tool.

\subsection{Selecting a group of stakeholders}
The visualization tool that the team intended to build has impact on different stakeholders. There are those stakeholders that will be directly affected by the tool and those who will be indirectly affected. We had a single group of stakeholders to interview at the company that contained different categories all grouped together, the categories can be found in the article of community tool box [STK ANALYSIS]. On referring to the article, it describes three different kinds of stakeholders which are primary stakeholders, secondary stakeholders and key stakeholders. Some of the advantages of interviewing the stakeholders include:  getting more ideas, obtaining different perspectives, contribution to us in to avoid misunderstanding of the problem, and also increases the chances of delivering a valuable output to the stakeholders. The primary stakeholders are the people who are directly interacting with Elektra, named beneficiaries and the students (us) who are designing the tool, named the target. The act of visualization can aid the beneficiaries in obtaining a simplified overview of a particular section of Elektra which in turn can provide a quick understanding of the needed components and the interactions between a component and another component. The secondary stakeholders are the ones who are indirectly affected by Elektra. They are the ones who may be affected by the decision made by the stakeholders who interacts directly with Elektra, which in turn will be due to the use of visualization tool on Elektra. Our supervisors at the university can also be in a group of secondary stakeholders, their are not directly interacting with the tool but they also play a greater role in pushing forward our goals for delivering the visualization tool. The key stakeholders are the one funding this project and these can be the two supervisors from the company. All these stakeholders play a very important role in bringing efforts that could result to a better solution of the problem that we intend to resolve.\\
We, who are students and as part of the primary stakeholders (targets), we intended to build a tool that can aid in quick understanding of the architecture and also provide the job opportunity for the need of having such a tool in overview of the architecture. We selected a group of stakeholders to interview. We explained them what were our intentions and what they could provide us to make sure we get the data that we needed.\todo{[to be filled in]}

\subsection{Set-up of goals and Interview questions}
It was very helpful to identify the goals of study. This had the purpose of understanding well the problem and focus on the things that are useful to our thesis. The article [MTDLG DATA] has explained couple of ways that we followed on resulting to the questions of interests used in the interview. The steps that has been applied for data collection as mentioned in the article were (1) :  Establishing the goals for data collection, (2) : Developing a list of questions of interests and (3) : Establishing data categories. 

\subsection{Identifying the needs of the stakeholders}
blah blah blah ... \todo{[to be filled in]}

\subsection{Extracting data from Elektra}
blah blah blah ... \todo{[to be filled in]}

%
\section{Analysis of data}
blah blah blah ... \todo{[to be filled in]}

\subsection{Discovery of metrics}
blah blah blah ... \todo{[to be filled in]}

\subsection{Identification of important components of electrical architectures}
blah blah blah ... \todo{[to be filled in]}

%
\section{Visualization of the important components of electrical architectures for stakeholders}
blah blah blah ... \todo{[to be filled in]}

\section{Evaluation of the visualization process}
blah blah blah ... \todo{[to be filled in]}