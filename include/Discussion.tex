% CREATED BY NATTAPON THATHONG, 2016
\chapter{Discussion}
\lettrine[findent=2pt]{\fbox{\textbf{T}}}{ }his chapter presents answers to the research questions stated in Section~\ref{IN:rq}.

\section{Answers to the research questions}
\setcounter{que}{0}

%%%%%%%%%
\begin{que}
What are the needs of stakeholders towards the visualization of the electrical architectures?
\end{que}

\vspace{1em}
In Section~\ref{IM:identifying_scope_visualization} we have mentioned that we arranged a meeting with Håkan Dahlen, a Software Developer, who works with one of the biggest and important ECUs called CEM. In the meeting, he explained that Elektra had been treated as a database storing data of the electrical artifacts such as LCs and ports. One of the problems he had faced when doing his work was that it was quite difficult to have a clear picture of the connection among ports of the LCs due to the lack of visualization of the tool. One had to look for a particular port in Elektra, unless he/she knew by experience which port had a connection to which. Having a visualization of the connections among ports would be beneficial to him and his department.\\

He also said that the electrical system stored in Elektra was big. Focusing only a sub-system and visualizing it would be a good start. 

\vspace{2em}

%%%%%%%%%
\begin{que}
What are the differences of the needs from each stakeholder?
\end{que}

\vspace{1em}
\todo{work in progress}
\vspace{2em}


%%%%%%%%%
\begin{que}
How do we identify the needs of stakeholders and come up with metrics that can help to identify important components in the electrical architectures?
\end{que}

\vspace{1em}
\todo{work in progress}
\vspace{2em}

%%%%%%%%%
\begin{que}
How do we complement Elektra to satisfy the needs of stakeholders?
\end{que}

\vspace{1em}
The way we can complement Elektra is to create an automated visualization of the electrical architectures. Creating a plug-in for Elektra or making a tool is possible. 
\vspace{2em}


%%%%%%%%%
\begin{que}
How does the automated visualization of the electrical architectures impact the way of working?
\end{que}

\vspace{1em}
\todo{work in progress}
\vspace{2em}