
\chapter{Validity Threats}

\section{Conclusion validity}
What could hinder us not to draw a correct conclusion from the treatment and the outcomes of the experiment.\\

Fishing: Some of the questions in the interview, we knew the answer, we were looking for a confirmation. Example : How do you find information in the Database?
Limited data, It would have been interesting if we did a test on different subsystem to be sure that our visualization would work on different subsystem.
Different stakeholders have different needs, what we obtained form the  stakeholders have a bit of similarities but that does not mean we might have get less or more needs. The more stakeholders the merrier.\\

Reliability of measures : Some of the questions that we asked during the interview were not answered correctly, meaning in the way we wished to be answered. This might have been caused by either a wrong way of formulating  a question or maybe the respondent did not understand the question.


\section{Internal validity}
Here we talk about threats that can affect the independent variables without the researcher’s knowledge.\\

History: One of the questions that we during the interviews was to ask a stakeholder about a specific task that he/she has been doing recently. The tasks of stakeholders can differ from time to time. Different tasks may have similar or different specifications even if the evaluation was done to the same stakeholders more than once, we cannot be assured that the needs that we have analyzed will always be the same for different tasks.\\

Instrumentation: In our visualization, we had applied the MDSE technique to get a diagram out of a text. And the text was transformed using a tool named JSON discoverer. The tool does the transformation of the JSON file to XML files. We cannot be sure that the tool may be 100 percent efficient but again, its human who has designed a tool and nobody is perfect and so, this can also be a threat to the final diagram that we get for the visualization. However, we manually verified on the final diagram to make sure all the components and all the data elements found in the original file were mapped correctly.\\

Selection: At VCG, different people uses the Database for different purposes. In the interviews, we had 3 people, two system designers, one of them is now a software developer and one system tester. These are few people out of many that use the Database. The needs that we obtained are not good enough which means we would have probably obtained more needs if we had interviews more stakeholders including the secondary stakeholders (Ones who do not use the Database).

\section{Construct validity}
This concerns generalizing the results based on the theory or concept behind it.\\

Inadequate pre-operational explication of constructs: We had experienced a bit of misunderstanding while interviewing one of the stakeholders. A question was not clear and the person explained a lot of things and many of the response were out of topic The purpose was to know how does the person find the tool then the person decided to compare the tool to a similar requirement handling tool that we had not used or seen.\\

Mono-operational bias: The visualization was done on a single subsystem with rather few number of components and signals. The results is expected to be different if the system was rather bigger than the one visualized. Also the visualization would have been different if it involved different artifacts such as requirements, LACs, SWCs or ECUs and this would vary for different subsystems.


\section{External validity}
Here we discuss about the conditions that limit us to generalize the results.\\

Interaction of selection and treatment: The needs that we obtained do not cover the entire population of stakeholders at VCG. The company is big and hence it has different people that are responsible for different tasks. Since the task was to visualize the data stored in the Database, then it would have been better if the stakeholders that we interviewed were of different levels, meaning primary stakeholders, secondary stakeholders and key stakeholders. Some stakeholders do not interact with the Database but that does not mean, they do not have an impact on the needs that we analyzed. This means that, the stakeholder that who does not interacts with the Database have an indirect impact on the results of the needs that we have obtained at the end.\\

Interaction of setting and treatment: In the visualization , we have applied one way to visualize the data which is by using MDSE. There is number of different ways that we could have applied, and the results might have been better than the one we have. Perhaps the best way is to use different tools when doing visualizing the data.

